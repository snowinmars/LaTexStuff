\documentclass[a4paper,14pt,russian]{article}

\usepackage[english, russian]{babel}
\usepackage[T2A]{fontenc}
\usepackage{cmap} % для кодировки шрифтов в pdf
\usepackage[utf8]{inputenc}

\sloppy

\usepackage{graphicx}
\graphicspath{{img/}}
\usepackage{amstext, amssymb, amsfonts, amsmath, amsthm}
\usepackage{indentfirst} % отделять первую строку раздела абзацным отступом тоже
\usepackage[usenames,dvipsnames]{color} % названия цветов
\usepackage{amssymb}
\usepackage{listings}
\usepackage{caption}
\bibliographystyle{unsrt}

\linespread{1.3} % полуторный интервал
%\renewcommand{\rmdefault}{ftm} % Times New Roman
\frenchspacing
\renewcommand\contentsname{Projects List} %%% renaming the Table of Contents

%%%%%%%%%%%%
% страницы
% \captionsetup{figurewithin=section}
\renewcommand{\thefigure}{\arabic{figure}}
\usepackage{fancyhdr}
\pagestyle{fancy}
\fancyhf{}
\fancyfoot[R]{\thepage}
\fancyfoot[L]{CC BY-SA}
\fancyheadoffset{0mm}
\fancyfootoffset{0mm}
\setlength{\headheight}{17pt}
\renewcommand{\headrulewidth}{0pt}
\renewcommand{\footrulewidth}{0pt}
\fancypagestyle{plain}{
\fancyhf{}
\rhead{\thepage}}
\setcounter{page}{2} % начать нумерацию страниц с №

%%%%%%%%%%%%%%%%%%%%%%%%%%%%
% Для правильного начертания буквы ангстрем %%%%
% При использовании гарнитуры ftm                  %%%%
%%%%%%%%%%%%%%%%%%%%%%%%%%%%
%\newcommand*{\OrigAA}{}
%\let\OrigAA\AA
%
%\renewcommand*{\AA}{%
  %{\fontfamily{ptm}%
  %\selectfont%
  %\OrigAA%
  %\selectfont}%
%}
%%%%%%%%%%%%%%%%%%%%%%%%%%%%%

%%%%%%%%%%%%
% картинки
% см habrahabr.ru/post/144648/
\usepackage[tableposition=top]{caption}
\usepackage{subcaption}
\DeclareCaptionLabelFormat{gostfigure}{Рисунок #2}
\DeclareCaptionLabelFormat{gosttable}{Таблица #2}
\DeclareCaptionLabelSeparator{gost}{~---~}
\captionsetup{labelsep=gost}
\captionsetup[figure]{labelformat=gostfigure}
\captionsetup[table]{labelformat=gosttable}
\renewcommand{\thesubfigure}{\asbuk{subfigure}}
\usepackage{wrapfig}

%%%%%%%%%%%%%
% геометрия
\usepackage{geometry}
\geometry{left=2.5cm}
\geometry{right=1.5cm}
\geometry{top=2.0cm}
\geometry{bottom=2.0cm}

\renewcommand{\AA}{\ensuremath{\mathring{A}}}

\setcounter{section}{0}

\begin{document}

\section {Цель работы}
Изучение методов повышения мощности излучения рубинового оптического квантового генератора, экспериментальное исследование рубинового оптического квантового генератора с насыщающимся фильтром, определение формы импульса излучения, импульсной мощности и энергии излучения.

\section {Литература}

\begin{enumerate}
\item  Квантовая электроника. Малая энциклопедия, изд. СЭ. М., 1969 г.
\item Рябцев Н.Г. Материалы квантовой электроники, изд. "Сов.радио", М., 1972 г.
\item Микаэлян А.Л., Тер-Микаэлян М.Л., Турков Ю.Г. Оптические генераторы на твёрдом теле, изд. "Сов.радио", М., 1972 г.
\item Страховский Г.М., Успенский А.В. Основы квантовой электроники, изд. "Высшая школа", М., 1973 г.
\end{enumerate}

\section {Теоретическая часть}

\subsection {Введение}

Один из путей развития лазерной техники непосредственно связан с повышением интенсивности (мощности) излучения оптических квантовых генераторов (ОКГ). Стремление получить высокие мощности излучения обусловлено не только потребностями современной техники. Использование лазеров в научных исследованиях позволило показать зависимость оптических свойств веществ от интенсивности проходящего через них излучения и обнаружить ряд неизвестных ранее явлений, например, просветление вещества, самофокусировка оптического луча, преобразование частоты оптического излучения и т.д. При больших интенсивностях оптического луча исчезает красная границе фотоэффекта. Используя лазеры, удаётся наблюдать фотоэлектронную эмиссию при энергии кванта света в шесть-семь раз меньшей, чем энергия ионизации атомов водорода фотокатода.

Все перечисленные эффекты объединяет их зависимость от интенсивности или полной мощности оптического излучения, а некоторые из них имеют характерные пороги по интенсивности. Повышение мощности оптического излучения облегчает их наблюдение и использование в практических целях.

Лазеры с повышенной мощностью нашли широкое применение в исследованиях по управляемым термоядерным реакциям. Чрезвычайно высокая плотность плазмы ($n \sim 10^{21} \Doteq 10^{23} sm^{-3}$ и выше), возникающей при воздействии на вещество излучения ОКГ, облегчает протекание термоядерных реакций, так как, согласно критерию Лоусена

\begin{eqnarray}
n \tau > 10^{14} sm^{-3} s
\end{eqnarray}

время удерживания $\tau$ плазмы, необходимое для развития ядерной реакции, обратно пропорционально плотности плазмы и в приведённом выше примере лазерной плазмы очень мало ($\tau \sim 10^{-9} s$). В некоторых экспериментах с использованием ОКГ зарегистрированы нейтроны, имеющие термоядерное происхождение.

В настоящее время наибольшую мощность излучения имеют твердотельные лазеры, использующие в качестве рабочих веществ ионы хрома $Cr^{+++}$ в корунде (рубиновых ОКГ) и ионы неодима $Nd^{+++}$ в стекле, иттрий--алюминиевом гранате и т.д.

Изучению основных особенностей работы твердотельного ОКГ на примере рубинового лазера и методов повышения мощности излучения посвящена настоящая работа.

\subsection {Особенности работы рубинового ОКГ в режиме свободной генерации}

Излучение рубинового ОКГ формируется в активном элементе 1 ($Al_2O_3 \cdot Cr^{+++}$), расположенном между зеркалами 2 и 3 открытого резонатора (рис. 1). Одно из зеркал полупрозрачно (коэффициент пропускания порядка 40\%-60\%), что необходимо для вывода излучения за пределы лазера.

Активным элементом рубинового ОКГ является кристалл корунда ($Al_2O_3$), в котором часть атомов алюминия замещена атомами хрома. Обычно используется бледно-розовый рубин с содержанием хрома около $0.05\%$, что соответствует концентрации ионов хрома $1.6 \cdot 10^{19} sm^{-3}$. Отметим, что изменение содержания хрома приводит и к изменению окраски рубина. Например, при содержании хрома $0.5\%$ окраска рубина ярко-красная. Если хрома более $8\%$, то цвет кристалла становится зелёным.

Используемые в лазерах рубиновые стержни имеют форму цилиндра длиной до $20 sm$ и диаметром около $0.3 \Doteq 1.5 sm$. Торцы стержня тщательно полируют и добиваются их параллельности с точностью до нескольких сеукнд.

На атомы $Cr$ в кристалле действует сильное электрическое полу, создаваемое атомами алюминия и кислорода. В то же время взаимодействие атомов хрома друг с другом очень слабо, так как концентрация $Cr$ в кристалле мала. Энергетический спектр атома $Cr$ в рубине соответствует спектру свободного атома $Cr$, помещённого в сильное электрическое полу кристалла. Структура уровней $Cr$ в рубине приведена на рис.2. Атом хрома имеет широкие полосы $\varepsilon_3$ и $\varepsilon_4$ энергетических состояний. Ширина их составляет около $800 sm^{-1}$.



Переход с уровней $\varepsilon_3$ и $\varepsilon_4$ на основной уровень $\varepsilon_1$ соответствует излучению фотонов в зелёном и голубом интервалах спектра. Значительная ширина уровней $\varepsilon_3$ и $\varepsilon_4$ связана с воздействием на атомы хрома неоднородного поля кристаллической решётки.

Уровень $\varepsilon_2$ ($\varepsilon_{2a}$ и $\varepsilon_{2b}$) является узким. Он расщеплён на два подуровня, различающиеся по энергии друг от друга на величину $20 sm^{-1}$. Уровень $\varepsilon_2$ является метастабильным и при комнатной температуре имеет время жизни $3.3 \cdot 10^{-3} s$.

Следует заметить, что основной уровень $\varepsilon_1$ в действительности имеет сложную структуру, особенности которой используются для работы парамагнитных квантовых усилителей и не существенны для работы лазера.

Если возбудить атом хрома, переведя его из основного состояния $\varepsilon_1$ в полосы $\varepsilon_3$ и $\varepsilon_4$ , возвращение его в основное энергетическое состояние может происходить через метастабильный уровень $\varepsilon_2$. Правилом отбора для спина запрещены излучательные переходы с уровней $\varepsilon_3$ и $\varepsilon_4$ на уровень $\varepsilon_2$. Однако, в результате теплового взаимодействия ионов с кристаллической решёткой между этими уровнями возможны безызлучательные переходы, вероятность которых велика. Поэтому возбуждённые ионы хрома быстро переходят на метастабильный уровень $\varepsilon_2$.

Переходы с уровня $\varepsilon_2$ на основной уровень $\varepsilon_1$ маловероятны. Их вероятность в $10^4$ раз меньше вероятности переходов с уровней $\varepsilon_3$ и $\varepsilon_4$ на уровень $\varepsilon_2$. Переходы $\varepsilon_2 \to \varepsilon_1$ сопровождаются люминесценцией на длинах волн $\lambda = 6934 \text{\AA}$ ($R_1$-линия) и $\lambda = 6929 \text{\AA}$ ($R_2$-линия), что обусловнело наличием двух подуровней $\varepsilon_{2a}$ и $\varepsilon_{2b}$. Длина волны и ширина этих линий заметно зависят от температуры. При температурах свыше $300 ^\circ K$ вследствие тепловых колебаний решётки линий $R_1$ и $R_2$ настолько расширяются, что происходит перекрытие их. С понижением температуры обе линии сильно сужаются и смещаются в коротковолновую область.

Не всё излучение рубина сосредоточено в $R$-линиях. Измерения показывают, что люминесценция в $R$-линиях составляет около $65\%$ общего излучения. $35\%$ излучения приходятся на линии, интенсивность которых быстро возрастает с повышением концентрации хрома. Их возникновение, вероятно, обусловлено взаимодействием ионов хрома.

При возбуждении ионов хрома большая часть переходов в основное энергетическое состояние происходит по рассмотренной выше схеме, т.е. с испусканием кванта света на $R$-линиях. Обычно среднее значение квантового выхода составляет $0.7$. Квантовый выход непосредственно в $R$-линиях равен $0.52$, что означает, что число спонтанно излучённых квантов света в $R$-линиях составляет $52\%$ от общего числа поглощённых квантов при возбуждении рубина.

Практически все ОКГ на рубине работают на линии $R_1$, так как вероятность перехода на ней больше, чем на линии $R_2$, и, следовательно, проще достигаются пороговые условия генерации.

Следует отметить также, что люминесценция рубина на $R$-линиях поляризована, например, степень поляризации излучения $R_1$-линии составляет $80\%$.

Перечисленные выше свойства рубина говорят о пригодности использования его в качестве рабочего вещества оптических квантовых генераторов. Суммируя вышесказанное, отметим, что

\begin{enumerate}
\item  Рубин имеет излучательный переход ($\varepsilon_2 \to \varepsilon_1$) в оптическом диапазоне длин волн, что позволяет использовать уровни $\varepsilon_2$ и $\varepsilon_1$ в качестве верхнего и нижнего уровней рабочего перехода ОКГ.
\item  Верхний уровень $\varepsilon_2$ в рабочем переходе является метастабильным, что облегчает получение избыточной населённости уровня $\varepsilon_2$.
\item  Рубин имеет широкие полосы поглощения $\varepsilon_3$ и $\varepsilon_4$, что позволяет использовать для его возбуждения мощные источники некогерентного излучения с широкой полосой излучаемых частот.
\item  Рубин имеет высокий квантовый выход, т.е. число излучаемых спонтанно квантов составляет значительную часть от числа поглощённых квантов при возбуждении рубина.
\item  Рубин допускает изготовление оптических однородных стержней, имеющих малые нерезонансные потери на частоте перехода $\nu_{21}$.
\end{enumerate}

В заключение отметим, что, как следует из рис.2, рубин как активный материал работает по трёхуровневой схеме.

\subsection {Принцип работы рубинового ОКГ}

В основе работы рубинового ОКГ лежит явление индуцированного излучения. Это явление, постулированное А. Эйнштейном в 1916 году, заключается в том, что при воздействии кванта света $h \nu$ на квантовую систему (например, на атом), имеющую энергетические уровни $\varepsilon_2$ и $\varepsilon_3$ с частотой перехода между ними $\nu_{21}$

\begin{eqnarray}
\nu_{21} = \cfrac {\varepsilon_2 - \varepsilon_1} {h}
\end{eqnarray}

и находящуюся в возбуждённом состоянии $\varepsilon_2$, последняя может вынужденно излучить (при $\nu = \nu_{21}$) квант света, тождественный падающему (т.е. их частоты, направления распространения, поляризация и т.д. неразличимы), переходя в нижнее энергетическое состояние. Вероятность этого процесса пропорциональна плотности падающего излучения и в отсутствие вырождения уровней $\varepsilon_1$ и $\varepsilon_2$ совпадает с вероятностью резонансного поглощения излучения квантовой системой, если она находится в нижнем энергетическом состоянии $\varepsilon_1$.

Таким образом, при прохождении излучения через вещество происходит (при $\nu \ \nu_{21}$)уменьшение его интенсивности за счёт взаимодействия с атомами, находящимися в нижнем энергетическом состоянии, и одновременное увеличение интенсивности за счёт вынужденного излучения атомами, находящимися в верхнем энергетическом состоянии. Суммарный эффект, естественно, будет определяться разностью населённостей уровней $n_2$ и $n_1$, т.е. разностью количества атомов в единице объёма, находящихся в верхнем $\varepsilon_2$ и нижнем $\varepsilon_1$ энергетических состояниях.

Полагая, что при каждом переходе атома из одного энергетического состояния в другое происходит излучение или поглощение одного кванта света, можно записать для изменения интенсивности излучения $dJ$ при прохождении слоя атомов толщиной $dx$:

\begin{eqnarray}
dJ = B_{21} J(n_2 - n_1) dx\;\;\;\;(1)
\end{eqnarray}
где
$\;\; W_{21} = B_{21} J$ - вероятность вынужденного излучения,
$\;\; B_{21}$ - коэффициент Эйнштейна
$\;\; n_i$ - населённость уровня $\varepsilon_i \;\; (i = 1, 2)$.

В обычных условиях при термодинамическом равновесии распределение населенностей атомных уровней подчиняется закону Больцмана:

\begin{eqnarray}
\cfrac {n_2} {n_1} = e^{- \cfrac {\varepsilon_2 - \varepsilon_1} {kT}}
\end{eqnarray}

Поэтому для равновесных состояний $dJ < 0$, т.е. по мере распространения излучения в среде, его интенсивность уменьшается.

Для того, чтобы излучение увеличивало интенсивность при прохождении через вещество необходимо, чтобы населенность верхнего уровня $n_2$ превышала населённость нижнего уровня $n_1$, т.е. необходимо выполнение неравенства $n_2 > n_1$. При выполнении $n_2 > n_1$ говорят о наличии в среде \underline{инверсной населённости} уровней. Таким образом, вопрос о возможности усиления и генерации света на основе эффекта индуцированного (вынужденного) излучения сводится к вопросу создания активной среды, т.е. возбуждённой среды с инверсной населённостью, способной отдавать свою избыточную внутреннюю энергию проходящему через неё излучению.

\subsection {Оптическая накачка}

В оптических квантовых генераторах на рубине возбуждение атомов осуществляется, как правило, с помощью \underline{оптической накачки}, т.е. воздействием на вещество световым излучением высокой интенсивности. Данный метод накачки облегчается из-за наличия у рубина широких полос поглощения (уровни $\varepsilon_3$ и $\varepsilon_4$ на рис.2a), что позволяет использовать для возбуждения ионов хрома большую часть некогерентного (и, следовательно, широкополосного) излучения пампы накачки, которую располагают рядом с рубином (см. рис. 1).

Нетрудно подсчитать минимальную мощность лампы, требуемую для создания инверсной населённости в рубине. Для выполнения условия $n_2 > n_1$ необходимо перевести на уровень $\varepsilon_2$ через уровни $\varepsilon_3$ и $\varepsilon_4$ по крайней мере половину ионов хрома, находящихся в нижнем энергетическом состоянии $\varepsilon_1$. Учитывая, что при комнатной температуре у рубина$\varepsilon_1 - \varepsilon_2 >> kT$, можно считать, что практически все ионы хрома при условии термодинамического равновесия находятся на уровне $\varepsilon_1$, т.е. $n_1 = 10^19 sm^{-3}$. На перевод одного атома $Cr$ на уровень $\varepsilon_2$ необходимо затратить энергию $\varepsilon_3 - \varepsilon_1 = 4 \cdot 10^{-12} \cfrac {erg} {atom}$. Тогда минимальная энергия, необходимая для возбуждения рубина в единичном объёме равна

\begin{eqnarray}
\varepsilon = \cfrac {1} {2} n_1 (\varepsilon_3 - \varepsilon_1) = 2 \cdot 10^7 \cfrac {erg} {sm^3}.
\end{eqnarray}

Следует учитывать также тот факт, что перевод атомов хрома в энергетическое состояние $\varepsilon_2$ должен происходить за время $\tau$, равное или меньшее, чем время жизни атомов на уровне $\varepsilon_2$, т.е. $\tau \leqslant 3.4 \cdot 10^{-3} s$

Если переход атомов в возбужденное состояние будет происходить более медленно, то из-за спонтанных и релаксационных переходов на уровень $\varepsilon_1$ заметно уменьшается населённость $\varepsilon_2$, что затрудняет реализацию неравенства $n_2 > n_1$. Таким образом, для создания инверсной населённости в каждом кубическом сантиметре кристалла рубина должна поглощаться энергия накачки $\sim 2 \cdot 10^7 erg$ за время $\sim 10^{-3} s$. Это означает, что поглощаемая мощность составляет $\sim 2 kW$ на $1 sm^3$. При объём кристалла в $10 sm^3$ необходимая мощность оптической накачки равно $20 kW$. следует заметить, что коэффициент использования световой энергии лампы накачки невелик. Даже при наличии у рубина широких полос поглощения используется для возбуждения атомов хрома лишь $10\% \Doteq 15\%$ свет лампы накачки. Поэтому в рассмотренном примере полная мощность лампы должна быть $\sim 200 kW$. В настоящее время такие мощности легко достигаются в газоразрядных лампах (например, с ксеноновым заполнением) в импульсном режиме.

Таким образом, при освещении рубина достаточно мощной лампой накачки происходит увеличение числа атомов на уровне $\varepsilon_2$ до величин, при которых $n_2 > n_1$. Рубин становится активной средой, способной усиливать проходящее излучение на частоту перехода $\nu_{21}$.

\subsection {Условия возникновения когерентного излучения в ОКГ}

Первичные кванты излучения на этой частоте возникают за счёт спонтанных переходов $\varepsilon_2 \to \varepsilon_1$. Их направление движение произвольно. Однако расположение рубина между зеркалами 2 и 3 (рис.1) создаёт преимущественное усиление излучения вдоль оси кристалла (перпендикулярно плоскости зеркал). Действительно, кванты света, излучаемые спонтанно под углом $\theta \ne 0$ к оси кристалла, усиливаются незначительно, так как мала длина пути их в кристалле. Эти кванты покидают рубин, практически не вызывая индуцированное излучение.

Кванты света, спонтанно излучаемые в направлении оси кристалла после прохождения вещества и отражение от зеркала возвращаются в вещество. Время пребывания их в веществе значительно. При каждом прохождении через рубин число квантов увеличивается за счет индуцированных переходов, сначала незначительно, так как вероятность вынужденных переходов пропорциональна плотности излучения, однако, по мере роста числа квантов увеличивается вероятность вынужденного излучения. Это приводит к формированию когерентного излучения в направлении оси кристалла.

Следует заметить, что в оптических квантовых генераторах, как и во всяких других генераторах имеется ярко выраженная положительная обратная связь, создаваемая зеркалами, которые возвращают излучение в вещество, где оно возникло.

Исходя из рассматриваемой модели ОКГ нетрудно сформулировать условия возникновения колебаний в лазере. Для этого проследим путь фотонов, которые в определённый момент времени находилась в сечении $a a'$ и распространяются к зеркалу 2, имеющему коэффициент отражения в $100\%$. Очевидно, что в лазере реализуется режим стационарных колебаний (а при малом количестве фотонов в плоскости $a a'$ - режим возникновения колебаний, т.е. пусковой режим), если число фотонов в плоскости $a a'$ после прохождения пути $a a' \to mirror 2 \to a a' \to mirror 3 \to a a'$ не изменится, иными словами, если увеличение числа фотонов при двукратном прохождении вещества за счёт вынужденных переходов будет скомпенсировано потерей их в веществе и в открытом резонаторе, образованном зеркалами 2 и 3.

Потери числа квантов в ОКГ могут быть обусловлены различными причинами:
\begin{enumerate}
\item Потери на излучение связаны с выводом излучения за пределы лазера через полупрозрачное зеркало 3
\item Нерезонансные потери в веществе, связанные с недостаточной оптической однородностью рубина и, следовательно, с рассеянием излучения на неоднородностях. К нерезонансным потерям можно отнести и омические потери в веществе.
\item Дифракционные потери на зеркалах, связанные с дифракцией лазерного излучения на зеркалах, имеющих конечные размеры.
\end{enumerate}

Дифракционные потери и потери на излучение зависят от геометрических размеров открытого резонатора и отражающих свойств зеркал и обычно определяют добротность открытого резонатора, которая имеет величины до $10^5 \Doteq 10^6$, что говорит о малой величине данного вида потерь.

Нерезонансные потери определяются качеством изготовления лазерного рубина и также имеют, как правило, малую величину.

Сказанное выше свидетельствует о том, что генерация в лазере начинает развиваться уже при малой инверсной населённости уровней $\varepsilon_2$ и $\varepsilon_1$, т.е. при $n_2 > n_1$.

Основные свойства излучения рубинового ОКГ в режиме свободной генерации

Генерация в рубиновом лазере, возникающая при малой инверсной населённости $n_2 \leqslant n_1$, препятствует увеличению населённости уровня $\varepsilon_2$ при продолжающемся воздействии лампы накачки, так как процесс сброса атомов хрома с уровня $\varepsilon_2$ в основное состояние при индуцированном излучении протекает весьма интенсивно. Поэтому на протяжении всего времени воздействия импульса накачки разность населённостей уровней $\varepsilon_2$ и $\varepsilon_1$ $\Delta n = n_2 - n_1$ лишь незначительно превышает пороговое значение $\Delta n_{doorstep}$, при котором возникает генерация. Данный режим работы рубинового ОКГ известен в литературу как режим свободной генерации. Характерные временные зависимости интенсивности излучения лампы накачки и лазера в режиме свободной генерации приведены на рис. 3.

Лампа накачки даёт импульс света длительностью $\sim 10^{-3} s$. В соответствии с этим рубиновый лазер излучает импульс когерентного света длительностью несколько меньшей, чем $10^{-3} s$. Последнее связано с тем, что нужно некоторое время для создания инверсии населённости, при которой начинается генерация. Генерация будет длиться до тех пор, пока интенсивность света лампы накачки не уменьшится до величины, при которой $\Delta n = \Delta n_{doorstep}$. Структура лазерного импульса сложна. Как правило, генерация представляет собой пульсацию излучения, которые имеют незатухающий характер, причём амплитуды "пичков" и расстояния между ними беспорядочно флуктуируют.

Такой режим излучения, обычно называемый "пичковым", хорошо наблюдается в рубиновом ОКГ при комнатной температуре. При охлаждение активного элемента до температуры жидкого азота хаотичность пульсации заметно уменьшается. До настоящего момента природа пичкового режима выяснена не полностью. По-видимому, в значительной степени она связана со свойствами оптического резонатора и, в частности, с наличием большого числа типов колебаний в нём. В отдельных работах указывается, что это явление обусловлено влиянием спонтанного излучения и физически связано с тем, что при большом количестве генерируемых типов колебаний энергия индуцированного излучения в каждом из них оказывается малой и сравнимой с энергией спонтанного излучения, в результате чего нестабильность отдельных типов колебаний не приводит к заметным изменениям суммарной интенсивности. Вероятно, определённый вклад в эффект вносит конечное время, необходимое для перевода атома в основное состояние по следующей схеме:

\begin{eqnarray}
\varepsilon_2 \to \varepsilon_3 (\varepsilon_4) \to \varepsilon_2 \to \varepsilon_1
\end{eqnarray}

Резкое уменьшение населённости уровня $\varepsilon_2$ при развитии генерации не может мгновенно скомпенсировать увеличением числа переходов $\varepsilon_3 \to \varepsilon_2$ при воздействии лампы накачки и в отдельные моменты времени $\Delta n = n_2 - n_1$ может значительно уменьшаться до величины порядка $\Delta n_{doorstep}$, что приводит к уменьшению интенсивности излучения. В пользу данного объяснения свидетельствует известный экспериментальный факт: пичковый режим тем более ярко выражен, чем больше время жизни метастабильного состояния. У лазеров, рабочее вещество которых имеет малое время жизни метастабильного состояния ($10^{-6} s$), режим хаотических пульсаций не наблюдается.

\subsection {Способы увеличения мощности излучения рубинового ОКГ}

Проблему увеличения мощности (интенсивности) излучения рубинового ОКГ, в принципе, можно решать двумя способами. Очевидно, что энергия излучения в режиме свободной генерации и, следовательно, мощность увеличатся, если увеличить длину рубинового стержня. При этом необходимо, естественно, увеличивать энергию светового излучения лампы накачки. Однако на практике этот способ не дал ощутимых результатов из-за трудностей при выращивании длинных рубиновых стержней для ОКГ и при решении проблемы охлаждения активных элементов в работающем лазере.

Более перспективным и плодотворным оказался второй способ, связанный с сокращением времени генерации в ОКГ при постоянной энергии накачки. Действительно, учитывая, что энергия лазерного излучения $E$ связана с мощностью $P$ отношением

\begin{eqnarray}
E = \int_0^t P dt
\end{eqnarray}
$\;\;$ где $t$ - длительность импульса излучения.

Нетрудно видеть, что мощность излучения тем больше, чем меньше $t$.

Как уже отмечалось выше, длительность импульса $t$ определяется тем промежутком времени, в течении которого сохраняется инверсная населённость уровней $\varepsilon_2$ и $\varepsilon_1$ $\Delta n = n_2 - n_1$, причём для $\Delta n$ растёт за счёт увеличения $n_2$ под воздействием лампы накачки и падает за счёт сброса атомов хрома со второго уровня на первый при вынужденном излучении. Таким образом, при постоянной энергии и длительности импульса света света лампы накачки тем меньше длительность импульса излучения ОКГ, чем быстрее сбрасываются атомы хрома с уровня $\varepsilon_2$. Количество переходов $\varepsilon_2 \to \varepsilon_1$ в объёме кристалла, равном $1 \times dx sm^3$, может быть легко подсчитано по формуле $(1)$, так как каждый такой переход сопровождается излучением одного кванта света. Разделив равенство $(1)$ на $h \nu_{21}$, имеем для числа вынужденных переходов $N$:

\begin{eqnarray}
N = B_{21} \; \cfrac {J} {h \nu_{21}} \; (n_2 - n_1) \;\;\;\; (2)
\end{eqnarray}

Из формулы $(2)$ видно, что число вынужденных переходов пропорционально разности населённостей $n_2 - n_1$. Если бы удалось перевести на уровень $\varepsilon_2$ большинство атомов хрома, оставив уровень $\varepsilon_1$ практически пустым, то согласно формулам $(2)$ и $(1)$ можно было бы значительно увеличить интенсивность изучения, резко сократив длительность его импульса. Расчёта показывают, что в рубиновом стержне длиной в $10 \Doteq 20 sm$ импульс излучения в ОКГ мог бы развиться а время $10^{-8} s$, при этом мощность излучения повышается до нескольких мегаватт.

С помощью существующих ламп накачки было бы легко "перебросить"  все атомы хрома на уровень $\varepsilon_2$, если бы в рубине не возникала генерация. Излучения лампы накачки легко переводит атомы $Cr$ в возбуждённое состояние до тез пор, пока е возникнет инверсия населённостей, т.е. превышение числа возбуждённых атомов над невозбуждёнными. Как уже отмечалось выше, генерация, начинающаяся сразу же после того, как чуть больше половины атомов $Cr$ переведено в возбуждённое состояние, сбрасывает атомы в основное состояние, препятствуя росту $\Delta n$, которое в режиме свободной генерации остаётся малой положительной величиной.

Перевод всех атомов в возбуждённое состояние можно осуществить, если на некоторое время убрать из системы обратную связь, т.е. отключить зеркала, устранив тем самым возможность генерации. После этого можно перевести атомы на уровень $\varepsilon_2$, на котором они будут находиться в течении $\sim 10^{-3} s$. Если за это время вновь включить зеркала, то произойдёт лавинообразный сброс атомов $Cr$ в основное состояние с излучением мощного короткого импульса. Процесс быстрого включения и выключения зеркала эквивалентен скачкообразному изменению добротности открытого резонатора лазера. Поэтому в литературе называют ОКГ, в которых генерация мощных импульсов осуществляется описанным выше способом, лазерами с модулированной или управляемой добротностью резонатора.

\subsection {Методы управления добротностью резонатора}

Для управления добротностьбю оптического резонатора применяются различные мезанические, лектрооптические и другие методы. Простейшая система с вращающимся диском изображдена на рис. 4

Диск имеет небольшое отверствие, которое при враении может совмещаться с фокусом двух линз в открытом рещонаторе. Когда лиск прерыват пучок света в фокуске лиез, добротность резонатора невелика и определяется коэффициентом отрадения левого зеркала, торца рубинового стержня и поверзности диска, которая обычно делается незеркальной.В промежуток времени, когда ищлучение проходич черезх отверстие, добротность резонатора резко возрастает за счёт отражения от второго зеркала. Сещуственным недостатком данной системы является сравнительно большое время переключения ($\sim 10^{-6} s$) добростноти, а также разрешение краёт отверстия диска при вывысоких уровнях мощности.

Более совершенная механческая система представлена на рис. 5. В ней перекбючение добротности осуществляется за счёт вращения одного из отрадателей (чаще всего - призмы с полным внутренним отражением, выполняющей роль непрозрачного зеркала), обращубщих оптический резонатор. Высокая добростность резонатора имеет место лишь в течении короткого интервала времени, когда отрадатели параллельны с большой степенью точности. При скоростях вращения $\sim 20000 .-. 30000 \cfrac {rot} {min}$, которые легко реализуются на практике, время переключения добротности составляет примерно $10^{-7} s$. Для увеличения скорости переключения в резонатор можно ввести дополнительно неподвижные отражатели (рис. 6). В этом случае зависимость добротности от угла повотора зеокада становится более сильной, и скорость переключения увеличивается вдвое.

Широкое применение для управление добротностью назодят различные электро- и магнитооптические затворы, основанные на использовании эффектов Керра, Поккельса, Фарадея. Принцип лдействия затвора с ячейкой Керра иллюстрирует рис. 7. Эффект Керра состоит в том, что под действием электрического поля вещество становится в оптическом отношении подобным одноосному кристаллу с отической осью вдоль направлоения электрического поля. Поэтому показатели преломдения для волны с электрическом вектором, параллельным придлженному полю, и для волны, перпендикулярной поляризации, оказываются различными. Разность хода, приобретаемая указанными волнами, пропорциональна квадрату напряжённости поля $E$ и длине ячейки $l$. Выражение для сдвига фаз может быть записано в виде


\begin{eqnarray}
\varphi = 2 \pi B l E^2
\end{eqnarray}
$\;\;$ где $B$ - постоянная Керра.

Эффект Керра имеет место во многих жидкостях и газах, особенно сльно он выражен в нитробензоле. Если, например, длина ячейкт с нитробензолом составляет $5 sm$, то для получения разности фаз $\cfrac {\pi} {2}$, нужно приложить электрическое поле напряжённостью $15000 \cfrac {V} {sm}$.

В схеме на рис. 7 поляризационные призмы $N_1$ и $N_2$ являются скрещёнными, т.е. они пропускают свет со взаимно перпендикулярной поляризацией. Главные плоскости поляризаторов составляют с направлением приложенного поля угол $45 ^\circ$. Если внешнее поле отсутствует, то система, образованная двумя поляризаторами и ячейкой Керра, полностью непрозрачна. При наложении поля вещество ячейки становится двоякопреломляющим, и свет, выходящий из неё, приобретает эллиптическую поляризация, так что часть излечения может пройти через оба поляризатора. Ксли веоличина приложенного напрядения или длина ячейки выбрана таким обращом, что разность хода лучей составляет половину длины полны, то выходящий свет будет иметь линей ную поляризацтя, перпендикулярную первоначальной, и прозрачность системы будет максимальной.

В ОКГ может использоваться более простая система с одним поляризатором (рис. 8). При этом параметры ячейки выбираюися иакими, чтобы сдвиг фах составляю $180 ^\circ$ при двукратном прохождении. Тогда затвор будет открыт при отсутствии поля и закрыт при наложении его. Слдедует заметить, что при использовании в качестве активной среды рубинас определённой ориентаций оси, при которой генерируетяс линейно поляризованое излучение, поляризатор в схеме рис. 8 может быть исключён. Достоинством электрооптических затворов с ячейкой Керра является малое время переключения, которое может составлять единицы наносекунд.

Другой возможный способ реализации поляризационных затворов, принцип действия которызх основан на изменении поляризации света, состоит в использовании эффекта Фарадея. эффект Фарадея заключается в том, что некоторые вешества, будучи помещёнными в магнитное поле, обладают способностью поверачивать плоскостьполяризации света, направление распространения которого совпадает с направлением приложенного поля. Угол поворота плоскости поляризации пропорционален наапряжённости магнитного поля $H$ и длина пути света в веществе $l$:


\begin{eqnarray}
\psi = \rho l H
\end{eqnarray}
$\;\;$ где $\rho$ - постоянная Верде.

Направление вращения определяется направлением приложенного поля и не зависит от направления распространения света. Величина $\rho$ может принимать значения $0.1 \cfrac {min} {sm ersted}$. Следует отметить, однако, что оптические затворы, использующие эффект Фарадея, широкого распространения не получили вследствие сравнительно больших потерь в управляющем элементе и недостаточно высокой скорости переключения (по сравнению с ячейкой Керра).

Все описанные затворы требуют значительного усложнения лектрической схемы дазера для синхронизации времени включения лампы накачки и перекюлючения добротности. Консрукция генератора с управляемой доборстностью можёт быть значительно упрощена, если использовать так называемые пассивные затворы, прозрачность которых меняется под действием светового излучения (рис. 9). Введение в резонатор пассивной ячейки, обладающей резонансными потерями, приводит к увеличению порогового уровня накачки, в резулоьтате чего к моменту начала генерации на метастабильном уровне накапливается большое число активных атомов. При возникновении генерации лазерное излучени, прозодящее через ячейку, резко уменьшает её потери, и запасённая энергия излучается в виде мозного испольса. Длительность этого испульса оказывается почти такой же, что и в редиме мгновенного включения добротности. В качестве веществ для пассивных щатровов используются, например, жидкие растворы фталоцианина, различные типы стёкол.

Для объяснения принципа действия пассивных ячеек поспользуемся следующей моделью фталоцианина. Схема основных энергетических уровней этого вещества представлена на рис. 9. Частота перехода между уровнями $\varepsilon_3$ и $\varepsilon_1$ находится в коасной области спектра и путём выбора подходящего растворителя можёет быть совмещена с частотой излучения рубинового ОКГ. Промежуточный уроенб $\varepsilon_2$ является метастабильным и имеет время жизни порядка $10^{-3} s$. Время безызлучательного перезода частиц с уровня $\varepsilon_3$ на уровень $\varepsilon_2$ составляет приблизительно $10^{-7} s$.

В нормальном состоянии все молекулы назодятся в основнойм энергетическом состоянии и способны поглощать излучение рубинового ОКГ. По мере поглощения энергии происходит увеличение числа возбуждённых молекул, и коэффициент поглощения падает. Данному эффекту благоприятствует то обстоятельство, что уровень $\varepsilon_2$ явояется матестабильным. Поэтому рабочее вещество ячкйки можно рассматривать как двухуровневую систему, в которой коэффициент поглощения излучения на частоте перезода $\varepsilon_3 \to \varepsilon_1$ определяется разностью населённостей этих уровней. Уведичегие населённости уровня $\varepsilon_3$ при поглощении энергии с одновременным уменьшением населённости уровня $\varepsilon_1$ приводит к резкому умеьншению коэффициента поглощения (насышение перезода $\varepsilon_3 \to \varepsilon_1$) в ячейк, т.е. к просветлению вещества, и добростность рещонатора возрастает, что обеспечивает получение короткого испульса излучения большой мощности.По аналогичной схемеработажют пассивные затворы, использующие азличные типы стёкол.

ИСпользование пассивных щатворов в ОКГ может приводить к появлению послеодвательнсоит мощных испольсов с высокой частотой повторения. После просветления вещества пассивного затрова и получения мощного испульса излучения рубинового ОКГ за счёт "сюброса" атомов хрома в основное состояние происходит обеднение уровня $\varepsilon_3$ вещества пассивного затрова ща счёт спонтанных перезордоа и увеличения населённсои уровня $\varepsilon_1$. Коэффициент поглощения излучения увеличивается, и добротность резонатора палает. Если к этому моменту времени продолжает действовать испульс оптической накачки, то может быть сформирован ешё один или несоклько испульсов излучения по рассмотренной выше схеме:

\begin{enumerate}
\item увеличение разности насеоённостей уровней $\varepsilon_2$ и $\varepsilon_1$ в рубине,
\item просветление вещества пассивного затрова,
\item излучение мощного испульса света рубиновым ОКГ.
\end{enumerate}

В обычных условиях расстояние мужду испульсами излучения составляет десятки микросекунд. Интервал между ними можно регулировать, изменяя уровень накачки и поглощение пассивного затвора.

/subsection{Основные характеристики ОКГ с управляемой добростностью}

Основные особенности рубинового ОКГ в редиме модуляции добротности могут быть изучены сравнительно просто, если предположить, что время "переключения" добротности равно нулю. В рамках данного предположения можно рассмотреть основыне зависимости энергетических и временных характеристик излучения от различных параметров резонатора и активной среды, а также оценить предельные значения длительности импульса, вызодной мощности и энергии.

/subusbsection {Мощность и энергия излучения}

При выводе уравнения ОКГ ограничимся одномерным приближением, учитывая высоую направленность лазерного излучения. Рассмотрим слой толщиной $\Delta x$ в активном веществе и составим уравнение непрерывности для интенсивности излучения. Число фотонов, проходящих в $1 s$ через $1 sm^2$ в положительном направлении оси $Ox$, обозначим через $J_1 (x, t)$, а в отрацательном - $J_2*x, t)$. Каждый элемент вещества характеризуется коэффициентом поглощения $\kappa$, который состоит из двух членов:

\begin{eqnarray}
\kappa = \beta - \sigma_{21} \Delta
\end{eqnarray}
$\;\;$ где $\sigma_{21} \ \cfrac {W_{21}} {J}$ - вероятность вынужденных перезодов при $J = 1.$

$\Delta$ - плотность инверсной населённости (см. ниже).

Первое слагаемое - $\beta$ - характеризует все потери, не связанные с резонансными эффектами индуцированного поглощения и испускания фотонов (нерезонансные потери). Второе слагаемое характеризует индуцированное усиление. Когда плотность инверсной населённости $\Delta = n_2 - n_1 > 0$ в $\beta$ мала, имеет место генерация фотонов активной средой, в результате чего интенсивности $J_1$ и $J_2$ увеличиваются.

Уравнения длояприращения интенсивности излучения на длине могут быть записаны в виде

\begin{eqnarray}
\left . \begin{matrix}
J_1 (x + \Delta x, t + \Delta t) - J_1(x, t) = (\sigma_{21} \Delta - \beta) J_1 (x, t) \Delta x \\
J_2 (x + \Delta x, t + \Delta t) - J_2(x, t) = (\sigma_{21} \Delta - \beta) J_2 (x, t) \Delta x
\end{matrix} \right.
\end{eqnarray}

Учитывая, что

\begin{eqnarray}
J_1(x + \Delta x, t + \Delta t) = J_1(x, t) + \cfrac {\partial J_1 (x + \Delta x, t)} {\partial t} \Delta t
\end{eqnarray}

получим, устремляя $\Delta x \to 0$:

\begin{eqnarray}
\left . \begin{matrix}
\cfrac {\partial J_1} {\partial x} + \cfrac {1} {v} \cfrac {\partial J_1} {\partial x} = (\sigma_{21} \Delta - \beta) J_1 \;\;\;\; (3) \\
\cfrac {\partial J_2} {\partial x} + \cfrac {1} {v} \cfrac {\partial J_2} {\partial x} = (\sigma_{21} \Delta - \beta) J_2 \;\;\;\; (4)
\end{matrix} \right.
\end{eqnarray}

Третье уравнение, связывающее плотность инверсной населённости $\Delta = n_2 - n_1$ с интенсивностями $J_1$ и $J_2$ можно получить, если учесть, что при генерации фотонов активной средой уменьшается разность населёностей уровней $\Delta = n_2 - n_1$. При получении уравнения будем пренебрегать влиянием спонтанного излучения на частоте $\nu_{21}$ накачки и на частоте релаксационныз процессов на изменение населённостей уровней $\varepsilon_1$ и $\varepsilon_2$, так как длительность лазерного импульса в редиме модуляции добротности мала. В слину сказанного выше

\begin{eqnarray}
- \cfrac {\partial n_2} {\partial t} = \cfrac {\partial n_1} {\partial t} = W_{21} (n_2 - n_1) \;\;\;\; (5)
\end{eqnarray}

Учитывая, что плотность энергии излучения $\rho(x, t)$ на частоте $\nu_{21}$ равна

\begin{eqnarray}
\rho = \cfrac {h \nu_{21}} {v} \Big( J_1 (x, t) + J_2 (x, t) \Big)
\end{eqnarray}

и, следовательно, $W_{21} = \sigma_{21} (J_1 + J_2)$ найдём из $(5)$:

\begin{eqnarray}
\cfrac {\partial \Delta} {\partial t} = -2 \Delta \sigma_{21} (J_1 + J_2) \;\;\;\;(6)
\end{eqnarray}

Приведённые уравнения должны быть дополены начальным и играничными условиями. Если длина активного образца $l$, а коэффициенты отражения плоских зеркал резонатора при $x = 0$ и $x = l$ равны соответственно $r_1$ и $r_2$, то граничные условия для генератора могут быть записаны в виде

\begin{eqnarray}
J_2(l, t) = r_2 J_1(l, t) \;\; ; \;\;\;\;\; J_1(0, t) = r_1 J_2(0, t)
\end{eqnarray}

В начальный момент времени $t = 0$ должно быть задано аспределение населённостей и интенсивностей излучения вдоль образца. При равномерной накачке можно положить $\Delta(x, 0) = Const$. Величины $J_1(x, 0)$ и $J_2(x, 0)$ определяются случайными фотонами, возникающими в результате спонтанного излучения. Расчёты показывают, что конечные результаты практически не зависят от выбора малых величин $J_1(x, 0)$ и $J_2(x, 0)$.

Уравнения $(3)$, $(4)$ и $(6)$ с соответствующими граничными и начальными условиями являются достаточно сложными. ПОэтому их, как правило, упрощают, переходя к усреднённым уравнения. Складывая уравнения $(3)$ и $(4)$ и обозначая $J_+ = J_1 + J_2$, $J_- = J_1 - J_2$, получим

\begin{eqnarray}
\cfrac {\partial J_-} {\partial x} + \cfrac {1} {v} \cfrac {\partial J_+} {\partial t} = (\sigma_{21} \Delta - \beta) J_+ \;\;\;\; (7)
\end{eqnarray}

Усредним уравнение $(7)$ по длине образца, введя $\overline{J_+}$ соотношением


\begin{eqnarray}
\overline{J_+} = \cfrac {1} {e} \int_0^l J_+ (x, t) dx
\end{eqnarray}

Аналогично введём $\overline{\Delta}$.

Из уравнения $(7)$ имеем:

\begin{eqnarray}
\cfrac {1} {e} \Big( J_- (e) - J_- (0) \Big) + \cfrac {1} {v} \cfrac {\partial \overline{J_+} {\partial t}} = \sigma_{21} \overline{\Delta J_+} - \beta \overline{J_+} \;\;\;\; (8)
\end{eqnarray}

Бедум считать также, что изменения величин $J_+$ и $\Delta$ по длине образца малы по сравнению с из средними щеачениями. Тогда


\begin{eqnarray}
\overline{\Delta J_+} = \overline{\Big( \overline{\Delta} - \delta \Delta (x) \Big)} = \overline{\Delta} \overline{J_+} + \overline{\delta \Delta (x) \cdot \delta J_+(x)} \approx \overline{\Delta} \cdot \overline{J_+} \;\;\;\; (9)
\end{eqnarray}

Граничные условия преобразуются к виду

\begin{eqnarray}
\left . \begin{matrix}
J_-(0, t) = \cfrac {r_1 - 1} {r_1 + 1} \; J_+ (0, t) \\
J_-(e, t) = \cfrac {1 - r_2} {1 + r_2} \; J_+ (e, t) \;\;\;\; (10)
\end{matrix} \right.
\end{eqnarray}

Подставляя $(9)$ и $(10)$ в уравнение $(8)$ и полагая, что $J_+(e) \approx J_+(0) \approx \overline{J_+}$, получим:

\begin{eqnarray}
\cfrac {1} {v} \cfrac {\partial \overline{J_+}} {\partial t} = \sigma_{21} \overline{\Delta} \overline{J_+} - \Big( \beta + \cfrac {1 - r_1} {l (1 + r_1)} + \cfrac {1 + r_2} {l (1 + r_2)} \Big) \cdot \overline{J_+} \;\;\;\; (11)
\end{eqnarray}

Последние два члена в квадратной скобке характеризуют потери на излучение $\beta_{\Sigma}$. Более строгий учёт распределения $J_+$ и $\Delta$ вдоль образца даёт выражение


\begin{eqnarray}
\beta_{\Sigma} = \cfrac {1} {2l} \ln \cfrac {1} {r_1 r_2}
\end{eqnarray}

совпадающее при $r_{1, 2} \approx 1$ с выражением, входящим в $(11)$. Поэтому выпишем усреднённое уравнение $(8)$ в виде:

\begin{eqnarray}
\cfrac {1} {v} \cfrac {d \overline{\Delta}} {dt} = \sigma_{21} \overline{\Delta} \overline{J_+} - (\beta + \beta_{\Sigma}) \; \overline{J_+} \;\;\;\; (12)
\end{eqnarray}

Аналогично записывается усреднённое уравнение $(6)$:

\begin{eqnarray}
\cfrac {d \overline{\Delta}} {dt} = -2 \; \sigma_{21} \overline{\Delta} \overline{J_+} \;\;\;\; (13_)
\end{eqnarray}

Подставив $J_+$ из $(13)$ в правую часть уравнения $(12)$, получим

\begin{eqnarray}
\cfrac {1} {v} \cfrac {d J_+} {dt} = \Big( - \cfrac {1} {2} + \cfrac {\beta + \beta_{\Sigma}} {2 \sigma_{21} \Delta} \Big) \; \cfrac {\partial \Delta} {\partial t}
\end{eqnarray}

Здесь и ниже знак усреднения опущен.

Отсюда

\begin{eqnarray}
J_+(t) = J_+(0) + \Big( \Delta(0) - \Delta(t) \Big) \; \cfrac {v} {2} + \cfrac {v} {2 \sigma_{21}} (\beta + \beta_{\Sigma}) \ln \cfrac {\Delta(t)} {\Delta(0)} \;\;\;\; (14)
\end{eqnarray}

где $J_+(0)$ и $\Delta(0)$ - средняя плотность потока фотонов и инверсной населённости при $t = 0$.

Выходная мощность генератора $P_{\Sigma}$ может быть вычислена по формуле

\begin{eqnarray}
P_{\Sigma}(t) = h \nu_{21} V J_+(t) \beta_{\Sigma} \; = \; \cfrac {1} {2} h \nu_{21} S J_+(t) \ln \cfrac {1} {r_2} \;\;\;\; (15)
\end{eqnarray}

В формуле $(15)$ положено $r_1 = 1$ (для полностью отражающего зеркала), $S \cfrac {V} {l}$ - площадь поперечного сечения образца.

Из формулы $(15)$ с учётом $(14)$ можно найти пиковое значение мощности излучения. Полагая $\cfrac {\partial J_+} {\partial t} \ 0$, получим, подставляя найденную при этом величину $\Delta$ в уравнение $(14)$:

\begin{eqnarray}
P_{\Sigma_{max}} = \cfrac {1} {2} h \nu_{21} \Delta(0) S \ln \cfrac {1} {r_2} \bigg( 1 - \cfrac {\beta + \beta_{\Sigma}} {\sigma_{21} \Delta(0)} \Big( 1 + \ln \cfrac {\sigma_{21} \Delta(0)} {\beta + \beta_{\Sigma}} \Big) \bigg) \;\;\;\; (16)
\end{eqnarray}

Формула $(16)$ позволяет рассчитать зависимость максимальной мощности илучения от коэффициента отражения полупрозрачного зеркала и начальноей инверсной населённости $\Delta(0)$. Типичные зависимости представлены на рис. 10.

Расчёты показывают существование оптимального значегия $r_2$, зависящего от ачальной инверсной населённости $\Delta(0)$. Существование экстремального значения $P_{\Sigma_{max}}$ может быть объяснено и с физиченской точки зрения. Уменьшение коэффициента отражения полупрозрачного еркала приводит при больгих $r_2$ к увеличению доли энергии, выводимой из лазера. Однако, если $r_2$ мало, то его уменьшение рещко уменьшает долю излучения, возвращаемого в активное вещество, что, в свою очередь, уменьгает вероятность вынужденного излучения и энергии, ищлучаемой активным веществом. Из формулы $(14)$ конечную величину инверсной населённости $\Delta x$, которая устанавливается по окончании импульса излучения. Полагая $J_+(t) = 0$ и $J_+ (0) \approx 0$, получим

\begin{eqnarray}
\Delta_K = \Delta(0) + \cfrac {1} {\sigma_{21}} \; (\beta + \beta_{\Sigma}) \ln \cfrac {\Delta K} {Delta(0)} \; = \; \Delta(0) + \Delta_{doorstep} \ln \cfrac {\Delta_K} {\Delta (0)} \;\;\;\; (17)
\end{eqnarray}

В формуле $(17)$ $\Delta_{doorstep} = \cfrac {1} {\sigma_{21}} (\beta + \beta_{\Sigma}$ -пороговое значение инверсной населённости, которое можно найти из уравнения $(12)$, полагая в нём $J_+ = Const$. Из формулы $(17)$ можно найти энергию, излучённую в единичном объёме активного вещества:

\begin{eqnarray}
\varepsilon_{\Sigma}' = h \nu_{21} (\Delta (0) - \Delta_K)
\end{eqnarray}

Тогда для энергии излучённого импульса имеем

\begin{eqnarray}
\varepsilon_{\Sigma}' = \cfrac {h \nu_{21} S} {4 \Delta_{doorstep} \sigma_{21}} \Big( \Delta(0) - \Delta_K \Big) \ln \cfrac {1} {r_2} \;\;\;\; (18)
\end{eqnarray}

Зависимости энергии испульса оптического излучения от коэффициента отражения $r_2$ при раздичных уровнях энергии накачки приведены на рис. 11. Их графика видно, что для получения максимальной эергии испульса необходимо подобрать оптимальный коэффициент отражения $r_2$. Следует заметить также, что оптимальные значения $r_2$, при которых реализуются экстремальные значения $\varepsilon_{\Sigma}$ и $P_{\Sigma_{max}}$, отличаюстя друг от друга.

\subsubsection{Влияние длины оптического резонатора на выходную мощность}

При расчёте выходной мозности ОКГ мы предполагали, что отражаюзие покрытия оптического резонатора нанесены непосредственно на торцевые поверхности активного вещества. Однако на практике из-за необходимости помещения в резонатор модулирующей ячейки расстояние $L$ между зеркалами резонатора отличается от длины активного стержня $l$. В этом случае излучение, распространяющееся в резонаторе от одного зеркала к другому, усиливается в течении $\Delta t = \cfrac {l} {c} n$ ($n$ - показатель преломления активного вещества), а в течении $\Delta t'' = \cfrac {L - l} {c}$ его интенсивность не изменяется. Это приводит к уменьшению максимальной мощности излучения в $\kappa$ раз и к соответствующему увеличению длительности испульса излучения, причём

\begin{eqnarray}
\kappa = \cfrac {n l + L - l} {n l}
\end{eqnarray}

Энергия излучения при этом излучется незначительно.

\subsubsection{Схема экспериментальной установке и методика эксперимента}

Экспериментальная установка для исследования рубинового ОКГ в режиме модуляции добротности (рис. 12) состоит из трёх основных узлов:

\begin{enumerate}
\item генератор на рубине с блоками питания и управления;
\item аппаратура для измерения энергии излучения и наблюдения формы генерируемого испульса;
\item юстировочный лазер на смеси газов $He-Ne$ с блоком питания.
\end{enumerate}

Активный элемент лазера - рубин вместе с лампой накачки ИФП-ВСО закреплён в лазерно головке. Внутренняя поверзхность лазерной головки покрыта серебром и отполирована. Взаимное расположение рубинового стержня и лампы накачки обеспечивает оптимальное освещение активного вещества импульсом света лампы накачки. В головке укреплено также полностью отражающее зеркало ОКГ (коэффициент отражения $r \approx 1.0$). Лазерная головка закрыта металлическим кожузом, который заземляется при выполнении лабоаторной работы. Со стороны, противоподоженной полностью отражаюему щуркалу, в кожузе имеется отверстие, размеры которого превышают диаметр рубинового стержня, а центр примерно совпадает с остбю рубинового суржня. Лазерная головка и кожух закреплены на текстолитовой плите, зафиксиврованной на оптической скамье. На плите установлены также держатели выходного полупрозрачного зеркала и модулятора, допускаюзие их юстировку с помощью юстировочных винтов. Зеркала лазера изготовлены из оптического стекла и имеют цилиндрическую форму. На один из торцов цилиндра нанесено многослойное диэлектрическое покрытие. Толщины и количество слоёв подбирают при изготовлении заркал так, чтобы при нормальном падении излучения рубинового ОКГ ($\lambda = 6943 \AA$) на многослойное покрытие произошло частичное или полное отрадение без потери излучения в зеркале. Многослойность диэлектрического покрытия облегчает в силу явления интерференции получение заданного коэффициента отражения. В данной лабораториной работе исполльзуются интерференционные диэлектрические зкркала с коэффициентами отрадения $r_1 = 1.0$ и $r_2 = 0.4$. В держателе модулятора добротности может быть закреплоена с помощью пружинных зажимов стеклянная пластинка $KC-19$, а также кюветы с раствором фталоцианина, выполняюие в открытом резонаторе лазера роль пассивного затвора.

В лазерную головку с помощью высоковольтных экранированных проводов подводится напряжение питания лампы накачки (до $1 kV$) с накопительной ёмкости $C = 2044 \mu F$, расположенной в блоке питания лазера, а также высоковольтный испульс (до $15 kV$), обеспечиваюющий первичный пробой разрядного промежутка лампы ИФП-800. Это напряжение поджина прикладывается к лазерной головке, которая изолирована от заземлённого кожуха.

Блоки питания иуправления обесечивают заряд накопительной ёмкости до заданного напряения, подсоединение её к электродам лампы ИФМ-800, выработку испульса и поджига разряда и, в случае необзодимости,разряд ёмкости $C$ через баластное сопротивление. На лицевой панели блока управления расподожены вольтметр, показывающий напряжение на накопительной ёмкости, переключатель пределов напряжения (гурбо), ручка плавной установки напряжения в заданных пределах (точно) и клвиши управления. Клавиши "вкл" и "заряд" при нажатии с помощью механического приспособления фиксирубются в новом подожении. ОСтальные клавиши при отпускании возвращаются висходное положение. Клавиши "откл" и "разряд" предназначены лишь для возвращения в исходное положение клавиш "вкл" и "заряд" соответственно и никаких электрических переключений в схеме не производят. Клавиша "вспышка" мезанически связана с клавишей "заряд" таким образом, что при возврещении в исходное положение коавиша "заряд" также выбрасывается в исходное положение.

В исходном положении накопительная ёмкость $C$ отключена от схемы питания и закорочена небольшим разрядным сопротивлением. Отсутствие напяжения на ёмкости контролируется вольтметром на лицевой панели пульта управления.

При нажатии клавиши "заряд" (после включения прибора и его прогрева в течении $5-7$ минут) с помощью реле замыкается цепь питания ёмкости $C$, от которой отключается разрядное сопротивление. Начинается заряд ёмкости, которой сопровождается повышением напряжения на эоектродах лампы ИФП-800, расположенный в лаазерной головке. Рост напряжения контолируется по вольтметру на лицевой панели пульта уравлдения. Одновременно с подачей напряжения на накопительную ёмкость срабатывает световая сигнализация "Внимание. Высокое напряжение". Рост напряжения на накопительной ёмеости происходит до величины, задаваемой переменными сопротивлениями "грубо" и "плавно" на лицевой панелли. При достижении заданной величины напряжения в схеме срабаывает реле, разрывающее цепь питания ёмкости $C$ и включающее цепь испульса поджига разряда и лампу освещения клавиши "заряд". Если теперь нажать клавишу "вспышка", то в блоке управления сформируется короткий высоковольтный испульс, зажигающий разряд в лампе накачки. В ОКГ формируется испульс когерентного излучения, который может быть зафиксирован измерительной авваатурой установки. При отпускании клавиши "вспышка" происходит выбрасывание в исзодное положение клавиши "заряд".

Если после окончания заряда накопительной ёмкости надать клавишу "разряд", то к ёмкости подключится разрядное сопротивление, в рещультате чего напряжение на ёмкости уменьшится до нуля, что фиксируется вольтметром. Это позволяет разрядить накопительную ёскость $C$, не зажигая разряд в лампе.

При работе с рубиновым лазером следует иметь в виду, что отключение накопительной ёмкости от источника питания при окончании её исключает возможность регулирования установившегося напряжения на ёмкости ручками "грубо" и "плавно" на пульте блока управления. Поэтому изменение напряжения на ёмкости следует проводить после разряда ёмкости через сопротивление, учитывая, что перевод переключателя "грубо" в соседнее положение приводит к изменению напряжения заряда на $~ 50 V$. Это значение сооветствует также максимальному изменению напряженияпри вращении ручки "плавно".

Для определения энергии излучения ОКГ в лабораторной работе используется калориметриченсикй измеритель энергии ИКТ-1М. Фопма испульса излучения фиксируется на экране запоминающего рсциоорнпафа С1-29. Излучение ОКГ делится на две неравные части с помощью светоделительной пластинки, укреплённой на оптической скамиье. Чтобы исключить возможность прямого попадания излучения со стороны оператора, проводящего эксперимент, светодеительная пластинка закрыта экраном. Стойка, закрепляющая пластинку на оптической скамье допускает юстировку пластинки с помощью юстировочного устройства.

Большая часть излучения ОКГ прозодит через светоделительную пластинку и попадает в измерительную головку прибора ИКТ-1М, отъюстированную предлагаемым ниже способом (см. раздел "Методика эксперимента"). Незначительная часть излучения ОКГ, отражаясь от светоделительной пластинки, попадает на катод фотоэлемента ФЭК-1, укреплённого на стойке рядом с оптической скамьёй. Коаксиальный фотоэлемент ФЕК-1 имеет широкую полосу пропускаемых частот, что позводяет использовать его для регистрации коротких лазерных импульсов в режиме модуляции добротности. Для устранения паразитной засветки фотокатода и уменьшения амплитуды лазерного импульса до требуемой величины вход ФЭК закрыт пластинкой с большим коэффициентом поглощения в видимой области спектра частот. На пластинке отмечено положение фотокатода с помощью юстировочной метки. Совмещение с юстировочной меткой точки попадания луча юстировочного $He-Ne$ лазера, отражённого от светоделительной пластинки, расположенной под углом $45 ^\circ$ к оси юстировочного дахера и кристалла рубинового ОКГ гарантирует засветку фотокатода ФЭК лазерным испульсом. Фотоэлемент питается от источника постоянного напряжения через разделительный конденсатор, подключённый к коаксиальному выходу ФЭКа.

Испульс ищлучения, попадающий на вход ФЭКа, преобразуется посдедним в испульс тока, который поступает на вход запоминающего осциллогафа, позволяющего ищучать форму одиночныз импульсов. Запуск развёртки осциллографа осуществляется положительным импульсом, котоырй вырабатывается в блоке управлдения ОКГ и используется одновременно для зажигания разряда а лампе накачки. Продов синхронизации выведен из блока управления на взод синхронизации осциллографа. Записанный осциллографом испульс может быть затем воспроизведён на его экране в течении $2-3 min$. Так как скорость развёртки осциллографа калибрована, то по осциллограмме легко восстановить временнуб зависимость процесса излечения в лазере.

\subsubsection{Методика эксперимента. Юстировка резонатора рубинового ОКГ}

Параллельность зеркал резонатора и торцов убинового стержня может быть обеспечена их юстировкой с помощью гелий-неонового лазера ($\lambda = 6328 \AA$). Юстировочный лазер установлен на специальном юстировочном столике, закремлённом на оптической скамье. Луч юстировочного $He-Ne$ лазера, проходя через попупрозрачное зеркало, частично отражается от укащанных элементов в сторону юстировочного лазера. Поэтому на диафрагнме-экране, укреплённой на выходе $He-Ne$ лазера видны достаточно чёткие пятна, отрадённые от всех элеиентов резонатор. Посдедовательнок совмещение из с отверстием диафрагмы обеспечивает достаточную параллельность отрадаюзих элементов, так как после выполненияописанных выше операций каждый из гних перпендикуляреен оси $He-Ne$ лазера. В рамках данной дабораторной работы производится лишь юстировка выодного полупрозрачного зеркала. Поэтому может быть предложен следующий порядок работ при юстирокке лазера:


\begin{enumerate}
\item Отодвинуть шторку, закрываюзуб оптический тракт установки и убрать с оптической скамьи измерительную головку ИКТ-2М и светоделительную пластинку.
\item Убедиться, что блок питания $He-Ne$ лазера заземлён, и что на пути дуча отсутствуют предметы, которые могут отразить излечение дазера в сторону оператора, выполняющего работу. ВНИМАНИЕ: Все юстировочные работы и прочие, выполняемые при открытой шторке, запрещается проводить, сидя у экспериментальной установки, так как луч юстировочного $He-Ne$ и исследуемого рубинового лазера проходит примерно на уровне глаз сидящего оператора.
\item Включить питание $He-Ne$ лазера и прогреть прибор в течении 10-15 минут.
\item Ручку регулировки тока разряда установить в среднее положение и нажать на кнопку зажигания разряда. При зажигании разряда произойдёт отклонение стрелки миллиамперметра, контролирующего ток разряда 7-10 [mA]. Убедиться в наличии генерации $He-Ne$ лазера. ПРИМЕЧАНИЕ: Если разряд в $He-Ne$ лазере врзник, а генерация излучения отсутствует, то следует произвести прогревание прибора при токе разряда 10 [mA] до возникновения генерации.
\item Убедиться, что полностью отражающее зеркало и рубиновый стержень отъюстированы. Для этого следует закыть отверстие в кожухе лазерной головки (например, копировальной бумагой), и открывая его найти отражение от указанных элементов лууча на диафрагме-экране юстировочного лазера. Если их центры не совпадают с отверстием в экране, то следует пригласить лаборанта или преподавателя.
\item Получить полупрозрачное зеркаор, укрепить его в держателе так, чтобы торец зеркала с диэлектрическим покрытием располагался в сторону рубинового лазера ВНИММАНИЕ: Во время установки зеркала в держатель необходимо следить, чтобы отражённый от зеркала луч юстировочного лазера не мог попасть в студентов, находящихся в лаборатории.
\item Вращением юистровочных винтов держителя полупрозрачного зеркала добиться совмещения отрадённого от зеркала луча с отверстием в диафрагме-экране юстировочного лазера.
\end{enumerate}

\subsubsection{Юстировка элементов оптического тракта}

После проверки качества юстировки оптического резоатора необходимо определить положение светоделительной пластины на оптической скамбе, обеспечивающее попадание излучения рубинового ОКГ на фотокатод широкополосного фотодиода ФЭК. В лабораторной установке фотодиод ФЭК закрыт поглощающей свет крышкой, которая уменьшентинтенсивность излучения на фотокатоде до величины, обеспечивающей нормальную работу фотоэлемента и одновременно исключает возможность попажания некогерентного излучения (дневной свет и т.п.) на фотокатод. На урышке фотокатода нанесена метка, показывающее положение фотокатода.

При размещении светоделительной пластины на оптической скаибе луч юстировочного $He-Ne$ лазера проходит через пластинку и, отразившить частично от входного зеркала лазера, вновь попадает на светоделительную поластину со сторонеы рубинового лазера. Излдучение, отразившееся от двух поверхностей светоделииельной пластинки (примерно $10\% - 15\%$ от падающего) в сторону ФЭК даёт на крышке фотоэдемента два пятна. Перемещая свтоделительную поастинку вместе со стойкой, крепящей её, вдоль оптческой скамьи и пользуясь юстировочными винтами, следует совместить эти пятна с юстировочной меткой на крышке ФЭК. При этом необходимо контролировать угол наклона светоделительной пластинки и оси юстировочного и исследуемого лазеров. Угол наклона будет сосавлять $45 ^\circ$, если прямая, соединяющая два пятна, создаваемые лазерным лучом на крышке ФЭК с двумя отчётливо различимыми пятнами на оверхностях пластинки, прерпендикулярна оси юстировочного лазера (или направлению его луча). В последнюю очередь юстируется измерительная головка ИКТ-1М. Для этого достаточно совместить, перемещая головку на юстировочном столике, изображение луча $He-Ne$ ОКГ с юстировочной меткой на задней стенке головки.

\subsection{Здаание}

\begin{enumerate}
\item Используя приведёненные в описании данные об исследуемом ОКГ, рассчитать минимальное напряжение заряда накопительной ёмкости лапмы накачки, при котором возможна генерация.
\item Проверить заземление всех приборов, входящих в установку и включить ИКТ-1М длоя прогрева в соответствии с иснструкцией.
\item Отъюстировать открытый резонатор лазера. Проверить качество юстировки. Для этого
\begin{enumerate}
\item Установть на пути лазерного луча копировальную бумагу;
\item Закрыть оптический тракт и лазер шторками;
\item Включить питание исследуемого лазера, нажав до упора на клавишу "вкл." на пульте управления и прогруть установку в течении $2 - 3$ минут;
\item Выставить оучками "напряжение заряда", "грубо", "точно" напряжение $700 [V]$;
\item Нажать на клавишу "заряд" и зафиксировать напрядение на вольтметре;
\item произвести вспышку, нажав до упора на клавишу "вспышка". ПРИМЕЧАНИЕ: Если после вспышки напряжение на ёмкости будет нарастать, \underline{следует}, не дожидаясь полного заряда ёмкости, \underline{немедленно} нажать на клавишу "разряд" и разрядить ёмкость.
\item Убедившись в отсутствии напряжения на ёмкости, выключить установку, нажав на клавишу "откл.";
\item Отодвинуть шторку и убедиться, что на копировальной бумаге имеется след от лазерного луча (светлое пятно от выгоревшей краски). Если пятнеа нет, следует проверить качество юстировки. ВНИМЕНИЕ: Лазерная головка не имеет принудительного охлаждения, поэтому временной интервал между вспышками лампы накачки должен составлять не менее пяти минут.
\end{enumerate}
\item Отъюстировать элементы оптического тракта.
\item Закрыть оптиеский тракт шторками
\item Включить источник питания ФЭК и после прогрева в течении 10 минут включить высокое напряжение и установить его равным $1000 [V]$.
\item Включить прибор С1-29.
\item Подготовить к работе приборы ИКТ-1М и С1-29 в соответствии с инструкциями.
\item Снять зависимость энергии лазерного излучения от напряжения $U$ на накопительной ёмкости, меняя последнее через $50 [V]$ до $1000 [V]$. Снять осциллограммы наблюдаемых испульсов излучения.
\item Включить исследуемый лазер и установить пассивный затвор.
\item Щакрыть оптический тракт шторками.
\item Снять зависимость энергий излучения и количества импульсов излучения от напряжения $U$ на накопительной ёмкости при тех же значениях $U$, что и в пункте 9.
\item Рассчитать энергию импльса излучения, среднюю мощность излечения и превышение начальной инверсной населённости $\Delta(o)$ над пороговой.
\end{enumerate}

\subsection{Контрольные вопросы}

\begin{enumerate}
\item Объясните мезанизм возникновения колебаний в рубиновом лазере.
\item В чём заключается метод оптической накачи при создании инверсной населённости в рубине?
\item Каковы причины возникновения "пичковой" структуры ищлучения рубинового лазера?
\item Объясните мезанизм увеличения мощности излучения при увеличении начальной инверсной населённости в рубине.
\item Проведите сравнительное описание методов модуляции добротности.
\item Чем определяется временной интервал между вспышками при выполнении работы?
\item Расскажите схему расчёта превышегния начальной инверсной населённости над пороговой.
\end{enumerate}

\subsection{Численные данные}
Диаметр активного элетента - $6.5 [mm]$

Длина активного элемента - $80 [mm]$

$\sigma_{21} = 2.5 \cdot 10^{-20} [sm^2]$

Ёмкость накопительного конденстаротра - $2400 [\mu F]$

$r_2 = 40\%$

$r_{imp} = 50 [ns]$ // TODO imp?

$d$ луча - $2.5 [mm]$

\end{document}